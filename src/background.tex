\chapter{The home planet}

The home planet of the Tvellas people

By convention, we will 

\section{Stellar system}

The home star of the Tvellas people is an F9 main sequence yellow dwarf of around 0.9 solar masses.

\section{Home galaxy}

The home galaxy of the Tvellas people is the Large Magellanic Cloud (LMC). This galaxy is a 
satellite of the Milky Way, and together, they reside in the Local Group. The home star lies in
the outer arms

The night sky from the home planet is particularly spectacular: the Milky Way spans over 30 degrees 
of the sky, and conveniently, lies close enough to the ecliptic to be seen all around the world.
The Small Magellanic Cloud is also visible, though from a smaller portion of the home planet thanks
to its more northerly declination. Within the LMC, vast numbers of star-forming regions are readily
visible, including the Tarantula Nebula, the most active star-forming region in the Local Group.

\chapter{The Tvellas people}

\section{Technological progression}

Tvellas society has displayed a technological progression comparable to that of our own. The
majority of the population lives in developed urban centers with utilities such as electricity and
running water, as well as transportation available on all scales from cars to local transit to
regional rail to passenger aircraft. Rail networks in particular are highly developed; this is the
case throughout the planet as compared to Earth.

The dominant sources of energy, however, are not fossil fuels, as their home planet does not have
any sources of ancient organic carbon comparable to those of Earth. Electricity is primarily
produced though either nuclear power or renewable energy sources, such as solar power, wind 
turbines, hydroelectric power, and biomass. In particular, ethanol and biodiesels serve the role of
fossil fuels, but their usage is restricted to applications are electric power is not readily
available or convenient.

The Haber-Bosch process has been developed, and nitrogen is readily converted to ammonia, and then
to other forms available for uptake by crops. Phosphorus is also abundant and readily available.
The homeland of the Tvellas people is particularly rich in phosphorus, and has long served as one
of the primary exports from the region as a fertilizer.

Computer technology has also developed to a comparable extent, and most people own at least one
personal computer (this term is used broadly, and includes devices similar to smartphones). A global
internet is readily accessible through wired or wireless access points. Radio and television
broadcasts are abundant and easily received over the air, though the internet has become the primary
source for information.

\section{Culture}

\subsection{Social classes}

Although Tvellas society is fairly egalitarian, class distinctions still exist.

\subsection{Interpersonal communication}

Throughout the world, the Tvellas are known for their direct communication in social settings.
