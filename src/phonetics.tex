\chapter{Phonetics, phonology, and phonotactics}

\section{Phonemic inventory}

\subsection{Consonants}

The Tvellas consonant system distinguishes four broad places of articulation: labial, alveolar,
dorsal, and glottal -- as well as six manners of distinction: stop, fricative, lateral fricative,
approximant, and lateral approximant. Phonetically, affricates are also present, but these will be
analyzed as stop and fricative sequences.

\subsection{Vowels}

The Tvellas vowel system is relatively large, consisting of seven monophthongs and several
diphthongs.

There are no contrastive vowel length distinctions.

\section{Supraphonemic features}

\subsection{Tone}

The Tvellas tone system is of medium complexity with respect to what has been established for most
human languages. It is more complex than the pitch accents of Japanese or Swedish, but less so than
Mandarin or Vietnamese. The closest comparisons would be those of Shanghainese, Yoruba, or Navajo.

The Tvellas language has two tonemes, \textit{high tone} and \textit{low tone}. Generally, only the
high tone is marked in the romanization, with an acute accent: \textbf{\'{a}}. However, the tone
system only distinguishes between tone contours in entire words: there are no monosyllabic words
which exhibit a tone distinction.

That being said, tone contours within syllables are not contrastive. However, many syllables have
tone contours which arise from tone sandhi. This is especially evident in trisyllabic words whose
initial syllable differs in tone from the final syllable: in those words, the middle syllable
serves as a bridge between the low and high tone regardless of what value it

Historically, there were three tonemes, the two above as well as \textit{middle tone}, which in the
standard Tvellas dialect generally collapsed into the low tone, though some diachronic processes
affected the outcomes; this is most evident in bound morphemes such as case markers. In divergent
dialects, some syllables exhibit creaky voice, similar to the Danish stod or Latvian broken tone,
and this generally corresponds to the historical low tone. The distribution of this phenomenon has,
however, been affected by the loss of final glottal stops in some contexts.

\subsubsection{Tone classes for bound morphemes}

Tvellas bound morphemes, in particular inflectional markers, may be said to have \textit{neutral
tone.} These morphemes often arise from historic morphemes which contained the middle tone, 

\subsection{Stress}

Most Tvellas words are stressed on the first syllable. Stress is usually indicated by an increase in
syllable nucleus length and loudness.

Some exceptions to this rule exist. Although Tvellas has strongly head-final tendencies, some
prefixing morphemes are unstressed
