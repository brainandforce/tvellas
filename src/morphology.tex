\chapter{Inflectional morphology}

\section{Nominal morphology}

Tvellas nominal morphology

\subsection{Number}

Tvellas exhibits a number system that contains the commonly observed singular and plural
distinction, but plurality in Tvellas takes many forms. Depending on the noun that is being
described, and the intended meaning, different plural markers may be used. This system resembles the
classifiers of languages such as Mandarin, but the system is restricted in scope, and the morphemes
used to mark plurals are not free morphemes, but behave more like clitics. This is due to their 
origin as true classifier words in the proto-Tvellas language.

\begin{center}
    \begin{tabularx}{\columnwidth}{| l | X | l |}
        \hline
        \textbf{Plural marker} & \textbf{Usage} & \textbf{Examples} \\
        \hline \hline
        --tıä & Separate items not comprising a whole & lep : léptıä \\
        \hline
    \end{tabularx}
\end{center}

\subsection{Noun classes and grammatical gender}

In this descriptive grammar, we will draw the distinction between \textit{noun classes}, which refer
to shared patterns of declension exhibited by nouns, and \textit{grammatical gender}, which refers
to agreement between nouns and other words in sentences. As an example, we will refer to the Russian
language, which exhibits both phenomena: although Russian is said to have a grammatical gender
system with a masculine, feminine, and neuter gender, and this is exhibited in the agreement of
adjectives and verbs in the past tense with nouns. However, the declension classes in Russian do not
correspond directly with these grammatical genders: masculine nouns in particular may be part of any
declension class.

The Tvellas language has a large number of noun classes, but no grammatical gender, similar to many
Uralic languages (Finnish, Hungarian).

\subsection{Noun declension}

Now that the 

\subsection{Suppleted forms}

Many languages exhibit \textit{suppletion}: the partial replacement of 

\section{Verbal morphology}

By contrast, verbal morphology is significantly simpler.
